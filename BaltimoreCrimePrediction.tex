\documentclass[]{article}
\usepackage{lmodern}
\usepackage{amssymb,amsmath}
\usepackage{ifxetex,ifluatex}
\usepackage{fixltx2e} % provides \textsubscript
\ifnum 0\ifxetex 1\fi\ifluatex 1\fi=0 % if pdftex
  \usepackage[T1]{fontenc}
  \usepackage[utf8]{inputenc}
\else % if luatex or xelatex
  \ifxetex
    \usepackage{mathspec}
  \else
    \usepackage{fontspec}
  \fi
  \defaultfontfeatures{Ligatures=TeX,Scale=MatchLowercase}
\fi
% use upquote if available, for straight quotes in verbatim environments
\IfFileExists{upquote.sty}{\usepackage{upquote}}{}
% use microtype if available
\IfFileExists{microtype.sty}{%
\usepackage{microtype}
\UseMicrotypeSet[protrusion]{basicmath} % disable protrusion for tt fonts
}{}
\usepackage[margin=1in]{geometry}
\usepackage{hyperref}
\hypersetup{unicode=true,
            pdftitle={Baltimore Crime Prediction},
            pdfauthor={Xu Yang},
            pdfborder={0 0 0},
            breaklinks=true}
\urlstyle{same}  % don't use monospace font for urls
\usepackage{color}
\usepackage{fancyvrb}
\newcommand{\VerbBar}{|}
\newcommand{\VERB}{\Verb[commandchars=\\\{\}]}
\DefineVerbatimEnvironment{Highlighting}{Verbatim}{commandchars=\\\{\}}
% Add ',fontsize=\small' for more characters per line
\usepackage{framed}
\definecolor{shadecolor}{RGB}{248,248,248}
\newenvironment{Shaded}{\begin{snugshade}}{\end{snugshade}}
\newcommand{\KeywordTok}[1]{\textcolor[rgb]{0.13,0.29,0.53}{\textbf{#1}}}
\newcommand{\DataTypeTok}[1]{\textcolor[rgb]{0.13,0.29,0.53}{#1}}
\newcommand{\DecValTok}[1]{\textcolor[rgb]{0.00,0.00,0.81}{#1}}
\newcommand{\BaseNTok}[1]{\textcolor[rgb]{0.00,0.00,0.81}{#1}}
\newcommand{\FloatTok}[1]{\textcolor[rgb]{0.00,0.00,0.81}{#1}}
\newcommand{\ConstantTok}[1]{\textcolor[rgb]{0.00,0.00,0.00}{#1}}
\newcommand{\CharTok}[1]{\textcolor[rgb]{0.31,0.60,0.02}{#1}}
\newcommand{\SpecialCharTok}[1]{\textcolor[rgb]{0.00,0.00,0.00}{#1}}
\newcommand{\StringTok}[1]{\textcolor[rgb]{0.31,0.60,0.02}{#1}}
\newcommand{\VerbatimStringTok}[1]{\textcolor[rgb]{0.31,0.60,0.02}{#1}}
\newcommand{\SpecialStringTok}[1]{\textcolor[rgb]{0.31,0.60,0.02}{#1}}
\newcommand{\ImportTok}[1]{#1}
\newcommand{\CommentTok}[1]{\textcolor[rgb]{0.56,0.35,0.01}{\textit{#1}}}
\newcommand{\DocumentationTok}[1]{\textcolor[rgb]{0.56,0.35,0.01}{\textbf{\textit{#1}}}}
\newcommand{\AnnotationTok}[1]{\textcolor[rgb]{0.56,0.35,0.01}{\textbf{\textit{#1}}}}
\newcommand{\CommentVarTok}[1]{\textcolor[rgb]{0.56,0.35,0.01}{\textbf{\textit{#1}}}}
\newcommand{\OtherTok}[1]{\textcolor[rgb]{0.56,0.35,0.01}{#1}}
\newcommand{\FunctionTok}[1]{\textcolor[rgb]{0.00,0.00,0.00}{#1}}
\newcommand{\VariableTok}[1]{\textcolor[rgb]{0.00,0.00,0.00}{#1}}
\newcommand{\ControlFlowTok}[1]{\textcolor[rgb]{0.13,0.29,0.53}{\textbf{#1}}}
\newcommand{\OperatorTok}[1]{\textcolor[rgb]{0.81,0.36,0.00}{\textbf{#1}}}
\newcommand{\BuiltInTok}[1]{#1}
\newcommand{\ExtensionTok}[1]{#1}
\newcommand{\PreprocessorTok}[1]{\textcolor[rgb]{0.56,0.35,0.01}{\textit{#1}}}
\newcommand{\AttributeTok}[1]{\textcolor[rgb]{0.77,0.63,0.00}{#1}}
\newcommand{\RegionMarkerTok}[1]{#1}
\newcommand{\InformationTok}[1]{\textcolor[rgb]{0.56,0.35,0.01}{\textbf{\textit{#1}}}}
\newcommand{\WarningTok}[1]{\textcolor[rgb]{0.56,0.35,0.01}{\textbf{\textit{#1}}}}
\newcommand{\AlertTok}[1]{\textcolor[rgb]{0.94,0.16,0.16}{#1}}
\newcommand{\ErrorTok}[1]{\textcolor[rgb]{0.64,0.00,0.00}{\textbf{#1}}}
\newcommand{\NormalTok}[1]{#1}
\usepackage{graphicx,grffile}
\makeatletter
\def\maxwidth{\ifdim\Gin@nat@width>\linewidth\linewidth\else\Gin@nat@width\fi}
\def\maxheight{\ifdim\Gin@nat@height>\textheight\textheight\else\Gin@nat@height\fi}
\makeatother
% Scale images if necessary, so that they will not overflow the page
% margins by default, and it is still possible to overwrite the defaults
% using explicit options in \includegraphics[width, height, ...]{}
\setkeys{Gin}{width=\maxwidth,height=\maxheight,keepaspectratio}
\IfFileExists{parskip.sty}{%
\usepackage{parskip}
}{% else
\setlength{\parindent}{0pt}
\setlength{\parskip}{6pt plus 2pt minus 1pt}
}
\setlength{\emergencystretch}{3em}  % prevent overfull lines
\providecommand{\tightlist}{%
  \setlength{\itemsep}{0pt}\setlength{\parskip}{0pt}}
\setcounter{secnumdepth}{0}
% Redefines (sub)paragraphs to behave more like sections
\ifx\paragraph\undefined\else
\let\oldparagraph\paragraph
\renewcommand{\paragraph}[1]{\oldparagraph{#1}\mbox{}}
\fi
\ifx\subparagraph\undefined\else
\let\oldsubparagraph\subparagraph
\renewcommand{\subparagraph}[1]{\oldsubparagraph{#1}\mbox{}}
\fi

%%% Use protect on footnotes to avoid problems with footnotes in titles
\let\rmarkdownfootnote\footnote%
\def\footnote{\protect\rmarkdownfootnote}

%%% Change title format to be more compact
\usepackage{titling}

% Create subtitle command for use in maketitle
\newcommand{\subtitle}[1]{
  \posttitle{
    \begin{center}\large#1\end{center}
    }
}

\setlength{\droptitle}{-2em}

  \title{Baltimore Crime Prediction}
    \pretitle{\vspace{\droptitle}\centering\huge}
  \posttitle{\par}
    \author{Xu Yang}
    \preauthor{\centering\large\emph}
  \postauthor{\par}
    \date{}
    \predate{}\postdate{}
  

\begin{document}
\maketitle

{
\setcounter{tocdepth}{2}
\tableofcontents
}
\section{Introduction}\label{introduction}

Crimes are a common social problem affecting the quality of life and the
economic growth of a city. Crime rate is an important factor that
affects people's housing and traveling choices, for example: should I
move to this new city or move away? What places should I avoid when I
travel to a new place?

Baltimore is one of the few cities in the United States that is famous
for its high crime rates, ranking 7th among the 10 most dangerous cities
in 2017 according to Forbes. It is certainly a place that needs more
caution for new-coming people. But before they settle down in the city,
what kind of information could be useful to them in deciding where they
should avoid going, at what time. Thus I propose to use the crime
database to predict crime occurrences in Baltimore. The results could
not only help the decision-making process of common citizens, but could
also be potentially utilized by the local police department to more
efficiently allocate their limited resources.

Although crimes could occur everywhere, it is common that criminals
stick to their routines: they would generally commit a crime at similar
times in their familiar locations. Given this common pattern, I will be
able to use a data mining approach to predict the ``hotspots''
(locations and times) for different types of crimes in the near future.
I also want to test whether crime occurrences are linked with
environmental effects such as temperature, humidity, and precipitation
events. If so, my predictions could be more accurate with the help of
real-time weather data.

\section{Datasets}\label{datasets}

I collected the data from various online databases, including: 1)
Baltimore crime occurrence reports including crime time, crime data,
crime type, neighborhood, district etc. 2012 to recent from the website
of
\href{https://www.baltimorepolice.org/crime-stats/open-data}{Baltimore
Police Department}; 2) historical weather data scraped from \textbf{API}
on \href{https://openweathermap.org/}{OpenWeatherMap}, including hourly
temperature, humidity, precipitation, pressure, and weather events; and
3) future weather data got from the same API to predict crime chance.

\section{Load libraries and data}\label{load-libraries-and-data}

\subsection{Load libraries}\label{load-libraries}

\begin{Shaded}
\begin{Highlighting}[]
\KeywordTok{library}\NormalTok{(tidyverse) }
\KeywordTok{library}\NormalTok{(lubridate)}
\KeywordTok{library}\NormalTok{(caret)}
\KeywordTok{library}\NormalTok{(ranger)}
\KeywordTok{library}\NormalTok{(forcats)}
\KeywordTok{library}\NormalTok{(stringr)}
\KeywordTok{library}\NormalTok{(pROC)}
\KeywordTok{library}\NormalTok{(gbm)}
\end{Highlighting}
\end{Shaded}

\subsection{Load data}\label{load-data}

\begin{Shaded}
\begin{Highlighting}[]
\NormalTok{crime <-}\StringTok{ }\KeywordTok{read_csv}\NormalTok{(}\StringTok{'~/Google Drive/GithubProject/Shiny_Application/BPD_Part_1_Victim_Based_Crime_Data.csv'}\NormalTok{) }\OperatorTok\StringTok{ }
\StringTok{  }\KeywordTok{distinct}\NormalTok{() }\CommentTok{# get distinct rows of data with no duplicates}
\NormalTok{weather <-}\StringTok{ }\KeywordTok{read_csv}\NormalTok{(}\StringTok{'~/Google Drive/GithubProject/Shiny_Application/Weather.csv'}\NormalTok{) }
\end{Highlighting}
\end{Shaded}

\section{General info}\label{general-info}

\subsection{Weather Data}\label{weather-data}

\begin{verbatim}
## Dimensions: 51978 28
\end{verbatim}

\begin{verbatim}
## Observations: 51,978
## Variables: 28
## $ dt                  <dbl> 1349096400, 1349186400, 1349190000, 134919...
## $ dt_iso              <chr> "2012-10-01 13:00:00 +0000 UTC", "2012-10-...
## $ city_id             <dbl> 4347778, 4347778, 4347778, 4347778, 434777...
## $ city_name           <lgl> NA, NA, NA, NA, NA, NA, NA, NA, NA, NA, NA...
## $ lat                 <lgl> NA, NA, NA, NA, NA, NA, NA, NA, NA, NA, NA...
## $ lon                 <lgl> NA, NA, NA, NA, NA, NA, NA, NA, NA, NA, NA...
## $ temp                <dbl> 285.43, 288.10, 290.51, 291.95, 293.01, 29...
## $ temp_min            <dbl> 283.15, 285.37, 288.15, 290.15, 291.15, 29...
## $ temp_max            <dbl> 290.15, 291.15, 293.15, 294.26, 294.26, 29...
## $ pressure            <dbl> 1014, 1014, 1015, 1016, 1015, 1014, 1014, ...
## $ sea_level           <lgl> NA, NA, NA, NA, NA, NA, NA, NA, NA, NA, NA...
## $ grnd_level          <lgl> NA, NA, NA, NA, NA, NA, NA, NA, NA, NA, NA...
## $ humidity            <dbl> 82, 72, 52, 55, 49, 43, 52, 45, 56, 59, 63...
## $ wind_speed          <dbl> 0, 0, 0, 0, 0, 0, 1, 0, 2, 2, 1, 0, 0, 0, ...
## $ wind_deg            <dbl> 0, 0, 0, 0, 0, 0, 230, 0, 150, 160, 170, 0...
## $ rain_1h             <dbl> NA, NA, NA, NA, NA, NA, NA, NA, NA, NA, NA...
## $ rain_3h             <lgl> NA, NA, NA, NA, NA, NA, NA, NA, NA, NA, NA...
## $ rain_24h            <dbl> NA, NA, NA, NA, NA, NA, NA, NA, NA, NA, NA...
## $ rain_today          <dbl> NA, NA, NA, NA, NA, NA, NA, NA, NA, NA, NA...
## $ snow_1h             <lgl> NA, NA, NA, NA, NA, NA, NA, NA, NA, NA, NA...
## $ snow_3h             <lgl> NA, NA, NA, NA, NA, NA, NA, NA, NA, NA, NA...
## $ snow_24h            <lgl> NA, NA, NA, NA, NA, NA, NA, NA, NA, NA, NA...
## $ snow_today          <lgl> NA, NA, NA, NA, NA, NA, NA, NA, NA, NA, NA...
## $ clouds_all          <dbl> 75, 75, 40, 1, 1, 75, 1, 90, 90, 90, 90, 1...
## $ weather_id          <dbl> 803, 803, 802, 800, 800, 803, 800, 804, 80...
## $ weather_main        <chr> "Clouds", "Clouds", "Clouds", "Clear", "Cl...
## $ weather_description <chr> "broken clouds", "broken clouds", "scatter...
## $ weather_icon        <chr> "04d", "04d", "03d", "01d", "01d", "04d", ...
\end{verbatim}

\subsection{Crime Data}\label{crime-data}

\begin{verbatim}
## Dimensions: 329394 16
\end{verbatim}

\begin{verbatim}
## Observations: 329,394
## Variables: 16
## $ CrimeDate         <chr> "12/08/2018", "12/08/2018", "12/08/2018", "1...
## $ CrimeTime         <chr> "23:20:00", "23:00:00", "23:00:00", "22:50:0...
## $ CrimeCode         <chr> "4E", "6D", "6D", "7A", "4E", "3AF", "4E", "...
## $ Location          <chr> "100 S EUTAW ST", "900 S CATON AVE", "2600 H...
## $ Description       <chr> "COMMON ASSAULT", "LARCENY FROM AUTO", "LARC...
## $ `Inside/Outside`  <chr> "I", "O", "O", "O", "I", "O", "I", "O", "O",...
## $ Weapon            <chr> NA, NA, NA, NA, NA, "FIREARM", NA, "KNIFE", ...
## $ Post              <dbl> 113, 832, 232, 425, 842, 733, 612, 935, 211,...
## $ District          <chr> "CENTRAL", "SOUTHWESTERN", "SOUTHEASTERN", "...
## $ Neighborhood      <chr> "Downtown West", "Violetville", "Canton", "G...
## $ Longitude         <dbl> -76.62083, -76.67137, -76.57891, -76.54639, ...
## $ Latitude          <dbl> 39.28724, 39.27355, 39.28211, 39.34841, 39.2...
## $ `Location 1`      <chr> "(39.287240000000, -76.620830000000)", "(39....
## $ Premise           <chr> "HOTEL/MOTE", "PARKING LO", "STREET", "STREE...
## $ crimeCaseNumber   <lgl> NA, NA, NA, NA, NA, NA, NA, NA, NA, NA, NA, ...
## $ `Total Incidents` <dbl> 1, 1, 1, 1, 1, 1, 1, 1, 1, 1, 1, 1, 1, 1, 1,...
\end{verbatim}

\section{Data preprocessing}\label{data-preprocessing}

\subsection{Crime Data}\label{crime-data-1}

\subsubsection{Combine crime types}\label{combine-crime-types}

There are 15 types of crime: COMMON ASSAULT, LARCENY FROM AUTO, AUTO
THEFT, ROBBERY - STREET, AGG. ASSAULT, LARCENY, SHOOTING, ROBBERY -
COMMERCIAL, BURGLARY, ROBBERY - RESIDENCE, ROBBERY - CARJACKING,
HOMICIDE, ASSAULT BY THREAT, ARSON, RAPE.

I combine all the crime into 6 different big categories based on
\href{https://en.wikipedia.org/wiki/Crime}{wikipedia}

\begin{Shaded}
\begin{Highlighting}[]
\CommentTok{# define look up table for crime types, combine crime type into 6 big categories}
\NormalTok{lut <-}\StringTok{ }\KeywordTok{c}\NormalTok{(}\StringTok{"COMMON ASSAULT"}\NormalTok{ =}\StringTok{ "ASSAULT"}\NormalTok{,}
         \StringTok{"LARCENY FROM AUTO"}\NormalTok{ =}\StringTok{ "PROPERTY"}\NormalTok{,}
         \StringTok{"AGG. ASSAULT"}\NormalTok{ =}\StringTok{ "ASSAULT"}\NormalTok{,}
         \StringTok{"ROBBERY - STREET"}\NormalTok{ =}\StringTok{ "ROBBERY"}\NormalTok{,}
         \StringTok{"LARCENY"}\NormalTok{ =}\StringTok{ "PROPERTY"}\NormalTok{,}
         \StringTok{"ASSAULT BY THREAT"}\NormalTok{ =}\StringTok{ "ASSAULT"}\NormalTok{,}
         \StringTok{"ROBBERY - CARJACKING"}\NormalTok{ =}\StringTok{ "ROBBERY"}\NormalTok{,}
         \StringTok{"AUTO THEFT"}\NormalTok{ =}\StringTok{ "PROPERTY"}\NormalTok{,}
         \StringTok{"SHOOTING"}\NormalTok{ =}\StringTok{ "SHOOTING"}\NormalTok{,}
         \StringTok{"HOMICIDE"}\NormalTok{ =}\StringTok{ "HOMICIDE"}\NormalTok{,}
         \StringTok{"BURGLARY"}\NormalTok{ =}\StringTok{ "PROPERTY"}\NormalTok{,}
         \StringTok{"ROBBERY - COMMERCIAL"}\NormalTok{ =}\StringTok{ "ROBBERY"}\NormalTok{,}
         \StringTok{"ROBBERY - RESIDENCE"}\NormalTok{ =}\StringTok{ "ROBBERY"}\NormalTok{,}
         \StringTok{"RAPE"}\NormalTok{ =}\StringTok{ "RAPE"}\NormalTok{,}
         \StringTok{"ARSON"}\NormalTok{ =}\StringTok{ "PROPERTY"}
\NormalTok{)}
\NormalTok{crime}\OperatorTok{$}\NormalTok{Type <-}\StringTok{ }\NormalTok{lut[crime}\OperatorTok{$}\NormalTok{Description]}
\end{Highlighting}
\end{Shaded}

\subsubsection{Collapse crime count to daily
count}\label{collapse-crime-count-to-daily-count}

\begin{Shaded}
\begin{Highlighting}[]
\NormalTok{crime <-}\StringTok{ }\NormalTok{crime }\OperatorTok\StringTok{ }
\StringTok{  }\KeywordTok{group_by}\NormalTok{(CrimeDate,  Neighborhood, Type) }\OperatorTok\StringTok{ }
\StringTok{  }\KeywordTok{summarize}\NormalTok{(}\DataTypeTok{count =} \KeywordTok{n}\NormalTok{()) }\OperatorTok\StringTok{ }
\StringTok{  }\KeywordTok{ungroup}\NormalTok{()}
\end{Highlighting}
\end{Shaded}

\subsubsection{Calculate Month, Year, Day of Month, Day of
Week}\label{calculate-month-year-day-of-month-day-of-week}

\begin{Shaded}
\begin{Highlighting}[]
\NormalTok{crime_daily <-}\StringTok{ }\NormalTok{crime }\OperatorTok\StringTok{ }
\StringTok{  }\KeywordTok{mutate}\NormalTok{(}\DataTypeTok{CrimeDate =} \KeywordTok{mdy}\NormalTok{(CrimeDate), }\DataTypeTok{Month =} \KeywordTok{month}\NormalTok{(CrimeDate), }\DataTypeTok{Year =} \KeywordTok{year}\NormalTok{(CrimeDate),}
         \DataTypeTok{Day =} \KeywordTok{day}\NormalTok{(CrimeDate),  }
         \DataTypeTok{Weekday =} \KeywordTok{weekdays}\NormalTok{(CrimeDate, }\DataTypeTok{abbreviate=}\NormalTok{T), }\DataTypeTok{Week =} \KeywordTok{week}\NormalTok{(CrimeDate)) }\OperatorTok\StringTok{ }
\StringTok{  }\KeywordTok{mutate}\NormalTok{(}\DataTypeTok{Weekday =} \KeywordTok{factor}\NormalTok{(Weekday, }\DataTypeTok{levels =} \KeywordTok{c}\NormalTok{(}\StringTok{"Mon"}\NormalTok{,}\StringTok{"Tue"}\NormalTok{,}\StringTok{"Wed"}\NormalTok{,}\StringTok{"Thu"}\NormalTok{,}\StringTok{"Fri"}\NormalTok{,}\StringTok{"Sat"}\NormalTok{,}\StringTok{"Sun"}\NormalTok{)))}
\KeywordTok{names}\NormalTok{(crime_daily)[}\DecValTok{1}\NormalTok{]=}\StringTok{'Date'}
\end{Highlighting}
\end{Shaded}

\subsection{Weather Data}\label{weather-data-1}

\subsubsection{spread categorical value of weather\_description to
numeric
value}\label{spread-categorical-value-of-weather_description-to-numeric-value}

I select temp, temp\_min, temp\_max, pressure, humidity, wind\_speed,
clouds\_all, and weather\_description from the weather dataset, perform
time conversion (the timezone of weather dataset and crime dataset is
not the same), and then spread weather\_description to numerical value
in a wide format

\begin{Shaded}
\begin{Highlighting}[]
\NormalTok{weather <-}\StringTok{ }\NormalTok{weather }\OperatorTok\StringTok{ }
\StringTok{  }\KeywordTok{select}\NormalTok{(dt_iso,temp,temp_min,temp_max,pressure,humidity,wind_speed,clouds_all, weather_description) }\OperatorTok\StringTok{ }
\StringTok{  }\KeywordTok{mutate}\NormalTok{(}\DataTypeTok{count =} \DecValTok{1}\NormalTok{, }\DataTypeTok{weather_description =} \KeywordTok{str_replace_all}\NormalTok{(weather_description,}\StringTok{' '}\NormalTok{,}\StringTok{'_'}\NormalTok{)) }\OperatorTok\StringTok{ }
\StringTok{  }\KeywordTok{mutate}\NormalTok{(}\DataTypeTok{DateTime =} \KeywordTok{ymd_hms}\NormalTok{(dt_iso,}\DataTypeTok{tz =} \StringTok{"America/New_York"}\NormalTok{), }\DataTypeTok{Date =} \KeywordTok{date}\NormalTok{(DateTime)) }\OperatorTok\StringTok{ }
\StringTok{  }\KeywordTok{distinct}\NormalTok{(dt_iso,}\DataTypeTok{.keep_all =}\NormalTok{ T) }\OperatorTok\StringTok{ }
\StringTok{  }\KeywordTok{spread}\NormalTok{(weather_description, count, }\DataTypeTok{fill =} \DecValTok{0}\NormalTok{) }\OperatorTok\StringTok{ }
\StringTok{  }\KeywordTok{select}\NormalTok{(}\OperatorTok{-}\NormalTok{dt_iso)}
\end{Highlighting}
\end{Shaded}

\subsubsection{collapse hourly weather data to daily
data}\label{collapse-hourly-weather-data-to-daily-data}

\begin{Shaded}
\begin{Highlighting}[]
\NormalTok{weather_daily <-}\StringTok{ }\NormalTok{weather }\OperatorTok\StringTok{ }
\StringTok{  }\KeywordTok{group_by}\NormalTok{(Date) }\OperatorTok\StringTok{ }
\StringTok{  }\KeywordTok{summarize}\NormalTok{(}\DataTypeTok{temp =} \KeywordTok{mean}\NormalTok{(temp, }\DataTypeTok{na.rm =}\NormalTok{ T), }\DataTypeTok{temp_min =} \KeywordTok{min}\NormalTok{(temp_min,}\DataTypeTok{na.rm =}\NormalTok{ T), }\DataTypeTok{temp_max =} \KeywordTok{max}\NormalTok{(temp_max),}
            \DataTypeTok{pressure =} \KeywordTok{mean}\NormalTok{(pressure, }\DataTypeTok{na.rm =}\NormalTok{ T), }\DataTypeTok{humidity =} \KeywordTok{mean}\NormalTok{(humidity, }\DataTypeTok{na.rm =}\NormalTok{ T), }\DataTypeTok{wind_speed =} \KeywordTok{max}\NormalTok{(wind_speed, }\DataTypeTok{na.rm  =}\NormalTok{ T),}
            \DataTypeTok{clouds_all =} \KeywordTok{mean}\NormalTok{(clouds_all,}\DataTypeTok{na.rm=}\NormalTok{T),}\DataTypeTok{broken_clouds=}\KeywordTok{sum}\NormalTok{(broken_clouds),}\DataTypeTok{drizzle =} \KeywordTok{sum}\NormalTok{(drizzle),}
            \DataTypeTok{few_clouds =} \KeywordTok{sum}\NormalTok{(few_clouds), }\DataTypeTok{fog=}\KeywordTok{sum}\NormalTok{(fog), }\DataTypeTok{freezing_rain =} \KeywordTok{sum}\NormalTok{(freezing_rain),}\DataTypeTok{haze=}\KeywordTok{sum}\NormalTok{(haze),}
            \DataTypeTok{heavy_intensity_drizzle =} \KeywordTok{sum}\NormalTok{(heavy_intensity_drizzle),}\DataTypeTok{heavy_intensity_rain=}\KeywordTok{sum}\NormalTok{(heavy_intensity_rain),}
            \DataTypeTok{heavy_intensity_shower_rain=}\KeywordTok{sum}\NormalTok{(heavy_intensity_shower_rain),}\DataTypeTok{heavy_snow=}\KeywordTok{sum}\NormalTok{(heavy_snow),}
            \DataTypeTok{light_intensity_drizzle=}\KeywordTok{sum}\NormalTok{(light_intensity_drizzle),}\DataTypeTok{light_intensity_shower_rain=}\KeywordTok{sum}\NormalTok{(light_intensity_shower_rain),}
            \DataTypeTok{light_rain=}\KeywordTok{sum}\NormalTok{(light_rain),}\DataTypeTok{light_rain_and_snow=}\KeywordTok{sum}\NormalTok{(light_rain_and_snow),}\DataTypeTok{light_shower_snow=}\KeywordTok{sum}\NormalTok{(light_shower_snow),}
            \DataTypeTok{light_snow=}\KeywordTok{sum}\NormalTok{(light_snow),}\DataTypeTok{mist=}\KeywordTok{sum}\NormalTok{(mist),}\DataTypeTok{moderate_rain=}\KeywordTok{sum}\NormalTok{(moderate_rain),}\DataTypeTok{overcast_clouds=}\KeywordTok{sum}\NormalTok{(overcast_clouds),}
            \DataTypeTok{proximity_shower_rain=}\KeywordTok{sum}\NormalTok{(proximity_shower_rain),}\DataTypeTok{proximity_thunderstorm=}\KeywordTok{sum}\NormalTok{(proximity_thunderstorm),}
            \DataTypeTok{proximity_thunderstorm_with_rain=}\KeywordTok{sum}\NormalTok{(proximity_thunderstorm_with_rain),}\DataTypeTok{scattered_clouds=}\KeywordTok{sum}\NormalTok{(scattered_clouds),}
            \DataTypeTok{shower_rain=}\KeywordTok{sum}\NormalTok{(shower_rain),}\DataTypeTok{shower_snow=}\KeywordTok{sum}\NormalTok{(shower_snow),}\DataTypeTok{sky_is_clear=}\KeywordTok{sum}\NormalTok{(sky_is_clear}\OperatorTok{+}\NormalTok{Sky_is_Clear),}
            \DataTypeTok{smoke=}\KeywordTok{sum}\NormalTok{(smoke),}\DataTypeTok{snow=}\KeywordTok{sum}\NormalTok{(snow),}\DataTypeTok{SQUALLS=}\KeywordTok{sum}\NormalTok{(SQUALLS),}\DataTypeTok{thunderstorm=}\KeywordTok{sum}\NormalTok{(thunderstorm),}
            \DataTypeTok{thunderstorm_with_heavy_rain=}\KeywordTok{sum}\NormalTok{(thunderstorm_with_heavy_rain),}\DataTypeTok{thunderstorm_with_light_drizzle=}\KeywordTok{sum}\NormalTok{(thunderstorm_with_light_drizzle),}
            \DataTypeTok{thunderstorm_with_light_rain=}\KeywordTok{sum}\NormalTok{(thunderstorm_with_light_rain),}\DataTypeTok{thunderstorm_with_rain=}\KeywordTok{sum}\NormalTok{(thunderstorm_with_rain),}
            \DataTypeTok{very_heavy_rain=}\KeywordTok{sum}\NormalTok{(very_heavy_rain))}
\end{Highlighting}
\end{Shaded}

\subsection{Join crime data with weather
data}\label{join-crime-data-with-weather-data}

\begin{Shaded}
\begin{Highlighting}[]
\NormalTok{weather_crime_daily <-}\StringTok{ }\NormalTok{crime_daily }\OperatorTok\StringTok{ }
\StringTok{  }\KeywordTok{left_join}\NormalTok{(weather_daily) }\OperatorTok\StringTok{ }
\StringTok{  }\KeywordTok{filter}\NormalTok{(}\OperatorTok{!}\KeywordTok{is.na}\NormalTok{(temp))}
\end{Highlighting}
\end{Shaded}

\section{Explanatory Data Analysis
(EDA)}\label{explanatory-data-analysis-eda}

\subsection{crime count versus month}\label{crime-count-versus-month}

\begin{Shaded}
\begin{Highlighting}[]
\NormalTok{crime_daily }\OperatorTok\StringTok{ }
\StringTok{  }\KeywordTok{group_by}\NormalTok{(Month, Type) }\OperatorTok\StringTok{ }
\StringTok{  }\KeywordTok{summarize}\NormalTok{(}\DataTypeTok{count =} \KeywordTok{sum}\NormalTok{(count)) }\OperatorTok\StringTok{ }
\StringTok{  }\KeywordTok{ggplot}\NormalTok{(}\KeywordTok{aes}\NormalTok{(}\DataTypeTok{x =}\NormalTok{ Month, }\DataTypeTok{y =}\NormalTok{ count, }\DataTypeTok{col =}\NormalTok{ Type)) }\OperatorTok{+}
\StringTok{  }\KeywordTok{geom_point}\NormalTok{() }\OperatorTok{+}
\StringTok{  }\KeywordTok{geom_line}\NormalTok{() }\OperatorTok{+}
\StringTok{  }\KeywordTok{facet_wrap}\NormalTok{(.}\OperatorTok{~}\NormalTok{Type, }\DataTypeTok{scales =} \StringTok{'free_y'}\NormalTok{)}
\end{Highlighting}
\end{Shaded}

\includegraphics{BaltimoreCrimePrediction_files/figure-latex/unnamed-chunk-11-1.pdf}

We can see there's a general trend that crime count drops in February
and increases during summer. This may be related to temperature change.
But different crimes exhibit different patterns. We can definitely
include month of year as a variable to predict crime

\subsection{crime count versus day of
month}\label{crime-count-versus-day-of-month}

\begin{Shaded}
\begin{Highlighting}[]
\NormalTok{crime_daily }\OperatorTok\StringTok{ }
\StringTok{  }\KeywordTok{group_by}\NormalTok{(Year,Month,Day, Type) }\OperatorTok\StringTok{ }
\StringTok{  }\KeywordTok{summarize}\NormalTok{(}\DataTypeTok{count =} \KeywordTok{sum}\NormalTok{(count)) }\OperatorTok\StringTok{ }
\StringTok{  }\KeywordTok{ungroup}\NormalTok{() }\OperatorTok\StringTok{ }
\StringTok{  }\KeywordTok{group_by}\NormalTok{(Day, Type) }\OperatorTok\StringTok{ }
\StringTok{  }\KeywordTok{summarize}\NormalTok{(}\DataTypeTok{count =} \KeywordTok{mean}\NormalTok{(count)) }\OperatorTok\StringTok{ }
\StringTok{  }\KeywordTok{ggplot}\NormalTok{(}\KeywordTok{aes}\NormalTok{(}\DataTypeTok{x =}\NormalTok{ Day, }\DataTypeTok{y =}\NormalTok{ count, }\DataTypeTok{color =}\NormalTok{ Type)) }\OperatorTok{+}
\StringTok{  }\KeywordTok{geom_point}\NormalTok{() }\OperatorTok{+}
\StringTok{  }\KeywordTok{geom_line}\NormalTok{() }\OperatorTok{+}
\StringTok{  }\KeywordTok{facet_wrap}\NormalTok{(.}\OperatorTok{~}\NormalTok{Type, }\DataTypeTok{scales =} \StringTok{'free_y'}\NormalTok{)}
\end{Highlighting}
\end{Shaded}

\includegraphics{BaltimoreCrimePrediction_files/figure-latex/unnamed-chunk-12-1.pdf}

For assault and property, there's a dramatic increase of crime count on
the first day of the month. It's very different for different crime
types.

\subsection{crime count versus day of the
week}\label{crime-count-versus-day-of-the-week}

\begin{Shaded}
\begin{Highlighting}[]
\NormalTok{crime_daily }\OperatorTok\StringTok{ }
\StringTok{  }\KeywordTok{group_by}\NormalTok{(Weekday,Type) }\OperatorTok\StringTok{ }
\StringTok{  }\KeywordTok{summarize}\NormalTok{(}\DataTypeTok{count =} \KeywordTok{sum}\NormalTok{(count)) }\OperatorTok\StringTok{ }
\StringTok{  }\KeywordTok{ggplot}\NormalTok{(}\KeywordTok{aes}\NormalTok{(}\DataTypeTok{x =}\NormalTok{ Weekday, }\DataTypeTok{y =}\NormalTok{ count, }\DataTypeTok{col =}\NormalTok{ Type)) }\OperatorTok{+}
\StringTok{  }\KeywordTok{geom_point}\NormalTok{() }\OperatorTok{+}
\StringTok{  }\KeywordTok{geom_line}\NormalTok{(}\KeywordTok{aes}\NormalTok{(}\DataTypeTok{group =}\NormalTok{ Type)) }\OperatorTok{+}
\StringTok{  }\KeywordTok{facet_wrap}\NormalTok{(.}\OperatorTok{~}\NormalTok{Type, }\DataTypeTok{scales =} \StringTok{'free_y'}\NormalTok{) }\OperatorTok{+}
\StringTok{  }\KeywordTok{theme}\NormalTok{(}\DataTypeTok{axis.text.x =} \KeywordTok{element_text}\NormalTok{(}\DataTypeTok{angle =} \DecValTok{60}\NormalTok{, }\DataTypeTok{hjust =} \DecValTok{1}\NormalTok{))}
\end{Highlighting}
\end{Shaded}

\includegraphics{BaltimoreCrimePrediction_files/figure-latex/unnamed-chunk-13-1.pdf}

Homicide, robbery, and rape show a similar decreasing pattern during the
week, while assault peaks at weekend and property peaks at Friday and
drops at weekend. Shooting first drops until Thursday and then increases
again.

\subsection{Crime count versus
temperature}\label{crime-count-versus-temperature}

\begin{Shaded}
\begin{Highlighting}[]
\NormalTok{crime_daily }\OperatorTok\StringTok{ }
\StringTok{  }\KeywordTok{group_by}\NormalTok{(Date) }\OperatorTok\StringTok{ }
\StringTok{  }\KeywordTok{summarize}\NormalTok{(}\DataTypeTok{count =} \KeywordTok{sum}\NormalTok{(count)) }\OperatorTok\StringTok{ }
\StringTok{  }\KeywordTok{inner_join}\NormalTok{(weather_daily) }\OperatorTok\StringTok{ }
\StringTok{  }\KeywordTok{ggplot}\NormalTok{(}\KeywordTok{aes}\NormalTok{(}\DataTypeTok{x =}\NormalTok{ temp}\OperatorTok{-}\FloatTok{273.15}\NormalTok{, }\DataTypeTok{y =}\NormalTok{ count)) }\OperatorTok{+}
\StringTok{  }\KeywordTok{geom_point}\NormalTok{(}\DataTypeTok{alpha =} \FloatTok{0.5}\NormalTok{) }\OperatorTok{+}
\StringTok{  }\KeywordTok{geom_smooth}\NormalTok{(}\DataTypeTok{method =} \StringTok{'lm'}\NormalTok{) }\OperatorTok{+}
\StringTok{  }\KeywordTok{xlab}\NormalTok{(}\StringTok{'Air temperature (ºC)'}\NormalTok{)}
\end{Highlighting}
\end{Shaded}

\begin{verbatim}
## Joining, by = "Date"
\end{verbatim}

\includegraphics{BaltimoreCrimePrediction_files/figure-latex/unnamed-chunk-14-1.pdf}

There exists a linear positive relationship between crime count and air
temperature, which is to be expected.

\subsection{Crime count by
neighborhood}\label{crime-count-by-neighborhood}

Because there are a total of 279 neighborhoods, I'll only display the
top 10

\begin{Shaded}
\begin{Highlighting}[]
\NormalTok{crime_daily }\OperatorTok\StringTok{ }
\StringTok{  }\KeywordTok{group_by}\NormalTok{(Neighborhood) }\OperatorTok\StringTok{ }
\StringTok{  }\KeywordTok{summarize}\NormalTok{(}\DataTypeTok{total_count =} \KeywordTok{sum}\NormalTok{(count)) }\OperatorTok\StringTok{ }
\StringTok{  }\KeywordTok{arrange}\NormalTok{(}\KeywordTok{desc}\NormalTok{(total_count)) }\OperatorTok\StringTok{ }
\StringTok{  }\KeywordTok{filter}\NormalTok{(total_count }\OperatorTok{>}\StringTok{ }\DecValTok{3967}\NormalTok{) }\OperatorTok\StringTok{ }
\StringTok{  }\KeywordTok{ggplot}\NormalTok{(}\KeywordTok{aes}\NormalTok{(}\DataTypeTok{x =} \KeywordTok{fct_reorder}\NormalTok{(Neighborhood, total_count), }\DataTypeTok{y =}\NormalTok{ total_count)) }\OperatorTok{+}
\StringTok{  }\KeywordTok{geom_bar}\NormalTok{(}\DataTypeTok{stat =} \StringTok{'identity'}\NormalTok{) }\OperatorTok{+}
\StringTok{  }\KeywordTok{coord_flip}\NormalTok{() }\OperatorTok{+}
\StringTok{  }\KeywordTok{xlab}\NormalTok{(}\StringTok{'Neighborhood'}\NormalTok{)}
\end{Highlighting}
\end{Shaded}

\includegraphics{BaltimoreCrimePrediction_files/figure-latex/unnamed-chunk-15-1.pdf}

Not suprisingly downtown area is the most dangerous area.

\section{More data preprocessing}\label{more-data-preprocessing}

\subsection{Cluster Neighborhoods}\label{cluster-neighborhoods}

Because there are a total of 279 neighborhoods in Baltimore, if we
include all the neighborhoods in our model, the model would take forever
to fit. Furthermore, the cluster could tell some info about
neighborhoods similarity in Baltimore.

\begin{Shaded}
\begin{Highlighting}[]
\NormalTok{crime_by_neighborhood <-}\StringTok{ }\NormalTok{crime }\OperatorTok\StringTok{ }
\StringTok{  }\KeywordTok{group_by}\NormalTok{(Neighborhood,Type) }\OperatorTok\StringTok{ }
\StringTok{  }\KeywordTok{summarize}\NormalTok{(}\DataTypeTok{count =} \KeywordTok{n}\NormalTok{()) }\OperatorTok\StringTok{ }
\StringTok{  }\KeywordTok{drop_na}\NormalTok{() }\OperatorTok\StringTok{ }
\StringTok{  }\KeywordTok{spread}\NormalTok{(Type,count,}\DataTypeTok{fill =} \DecValTok{0}\NormalTok{) }\CommentTok{# format the table to wide format for clustering}
\NormalTok{scaled =}\StringTok{ }\KeywordTok{scale}\NormalTok{(crime_by_neighborhood[,}\DecValTok{2}\OperatorTok{:}\DecValTok{7}\NormalTok{]) }\CommentTok{# scale the data before performing clustering}
\NormalTok{wss =}\StringTok{ }\DecValTok{0} \CommentTok{# initialize within cluster sum of squared. We choose cluster number based on this.}
\ControlFlowTok{for}\NormalTok{ (i }\ControlFlowTok{in} \DecValTok{1}\OperatorTok{:}\DecValTok{20}\NormalTok{) \{}
\NormalTok{  km.out =}\StringTok{ }\KeywordTok{kmeans}\NormalTok{(scaled, }\DataTypeTok{centers =}\NormalTok{ i, }\DataTypeTok{nstart =} \DecValTok{20}\NormalTok{, }\DataTypeTok{iter.max =} \DecValTok{50}\NormalTok{)}
\NormalTok{  wss[i] =}\StringTok{ }\NormalTok{km.out}\OperatorTok{$}\NormalTok{tot.withinss}
\NormalTok{\}}
\KeywordTok{plot}\NormalTok{(wss)}
\end{Highlighting}
\end{Shaded}

\includegraphics{BaltimoreCrimePrediction_files/figure-latex/unnamed-chunk-16-1.pdf}

To me, 10 clusters should be enough.

Do it again with 10 clusters in mind

\begin{Shaded}
\begin{Highlighting}[]
\NormalTok{km.out =}\StringTok{ }\KeywordTok{kmeans}\NormalTok{(scaled, }\DataTypeTok{centers =} \DecValTok{10}\NormalTok{, }\DataTypeTok{nstart =} \DecValTok{20}\NormalTok{, }\DataTypeTok{iter.max =} \DecValTok{50}\NormalTok{)}
\NormalTok{crime_by_neighborhood}\OperatorTok{$}\NormalTok{cluster =}\StringTok{ }\NormalTok{km.out}\OperatorTok{$}\NormalTok{cluster}
\NormalTok{neighborhood_cluster <-}\StringTok{ }\NormalTok{crime_by_neighborhood }\OperatorTok\StringTok{ }
\StringTok{  }\KeywordTok{select}\NormalTok{(Neighborhood,cluster) }
\NormalTok{neighborhood_cluster}\OperatorTok{$}\NormalTok{cluster =}\StringTok{ }\KeywordTok{factor}\NormalTok{(neighborhood_cluster}\OperatorTok{$}\NormalTok{cluster)}
\end{Highlighting}
\end{Shaded}

\subsection{Prepare the dataset}\label{prepare-the-dataset}

Right now, the crime dataset only contains the neighborhood and date
when crime happens. But most of the time, there's no crime happening for
a certain neighborhood. We need to prepare the crime dataset to set
count to 0 when no certain crime happens

\begin{Shaded}
\begin{Highlighting}[]
\NormalTok{Neighborhood =}\StringTok{ }\KeywordTok{unique}\NormalTok{(crime_daily}\OperatorTok{$}\NormalTok{Neighborhood[}\OperatorTok{!}\KeywordTok{is.na}\NormalTok{(crime_daily}\OperatorTok{$}\NormalTok{Neighborhood)])}
\NormalTok{Type =}\StringTok{ }\KeywordTok{unique}\NormalTok{(crime_daily}\OperatorTok{$}\NormalTok{Type)}
\NormalTok{Date =}\StringTok{ }\KeywordTok{seq}\NormalTok{(}\KeywordTok{ymd}\NormalTok{(}\StringTok{'2012-01-01'}\NormalTok{),}\KeywordTok{ymd}\NormalTok{(}\StringTok{'2019-01-12'}\NormalTok{), }\DataTypeTok{by =} \StringTok{'1 day'}\NormalTok{)}
\NormalTok{prepare <-}\StringTok{ }\KeywordTok{expand.grid}\NormalTok{(}\DataTypeTok{Date =}\NormalTok{ Date,}\DataTypeTok{Neighborhood =}\NormalTok{ Neighborhood,}\DataTypeTok{Type =}\NormalTok{ Type) }\OperatorTok\StringTok{ }
\StringTok{  }\KeywordTok{mutate}\NormalTok{(}\DataTypeTok{Month =} \KeywordTok{month}\NormalTok{(Date), }\DataTypeTok{Year =} \KeywordTok{year}\NormalTok{(Date),}
         \DataTypeTok{Day =} \KeywordTok{day}\NormalTok{(Date),  }
         \DataTypeTok{Weekday =} \KeywordTok{weekdays}\NormalTok{(Date, }\DataTypeTok{abbreviate=}\NormalTok{T), }\DataTypeTok{Week =} \KeywordTok{week}\NormalTok{(Date)) }\OperatorTok\StringTok{ }
\StringTok{  }\KeywordTok{mutate}\NormalTok{(}\DataTypeTok{Weekday =} \KeywordTok{factor}\NormalTok{(Weekday, }\DataTypeTok{levels =} \KeywordTok{c}\NormalTok{(}\StringTok{"Mon"}\NormalTok{,}\StringTok{"Tue"}\NormalTok{,}\StringTok{"Wed"}\NormalTok{,}\StringTok{"Thu"}\NormalTok{,}\StringTok{"Fri"}\NormalTok{,}\StringTok{"Sat"}\NormalTok{,}\StringTok{"Sun"}\NormalTok{)))}
\NormalTok{crime_daily <-}\StringTok{ }\NormalTok{prepare }\OperatorTok\StringTok{ }
\StringTok{  }\KeywordTok{left_join}\NormalTok{(crime_daily)}
\NormalTok{crime_daily}\OperatorTok{$}\NormalTok{count[}\KeywordTok{is.na}\NormalTok{(crime_daily}\OperatorTok{$}\NormalTok{count)] =}\StringTok{ }\DecValTok{0}
\KeywordTok{table}\NormalTok{(crime_daily}\OperatorTok{$}\NormalTok{count)}
\end{Highlighting}
\end{Shaded}

\begin{verbatim}
## 
##       0       1       2       3       4       5       6       7       8 
## 4026377  209021   37560    8536    2434     752     253      93      34 
##       9      10      11      12      15 
##      20       3       5       3       1
\end{verbatim}

\begin{Shaded}
\begin{Highlighting}[]
\NormalTok{proportion =}\StringTok{ }\NormalTok{crime_daily }\OperatorTok\StringTok{ }
\StringTok{  }\KeywordTok{group_by}\NormalTok{(Type) }\OperatorTok\StringTok{ }
\StringTok{  }\KeywordTok{summarize}\NormalTok{(}\DataTypeTok{prop =} \KeywordTok{sum}\NormalTok{(count }\OperatorTok{!=}\StringTok{ }\DecValTok{0}\NormalTok{)}\OperatorTok{/}\KeywordTok{n}\NormalTok{())}
\NormalTok{proportion}
\end{Highlighting}
\end{Shaded}

\begin{verbatim}
## # A tibble: 6 x 2
##   Type        prop
##   <chr>      <dbl>
## 1 ASSAULT  0.103  
## 2 HOMICIDE 0.00267
## 3 PROPERTY 0.209  
## 4 RAPE     0.00293
## 5 ROBBERY  0.0401 
## 6 SHOOTING 0.00458
\end{verbatim}

Because the proporation the crime happens is around 20\% even for
property crime, which has the highest occurances, I will turn this into
a \textbf{classification} task

\subsection{change crime to No or Yes based on
count}\label{change-crime-to-no-or-yes-based-on-count}

\begin{Shaded}
\begin{Highlighting}[]
\NormalTok{crime_daily}\OperatorTok{$}\NormalTok{Crime =}\StringTok{ }\KeywordTok{if_else}\NormalTok{(crime_daily}\OperatorTok{$}\NormalTok{count }\OperatorTok{==}\StringTok{ }\DecValTok{0}\NormalTok{, }\StringTok{'No'}\NormalTok{, }\StringTok{'Yes'}\NormalTok{)}
\end{Highlighting}
\end{Shaded}

\subsection{Join weather dataset and cluster info to crime
dataset}\label{join-weather-dataset-and-cluster-info-to-crime-dataset}

\begin{Shaded}
\begin{Highlighting}[]
\NormalTok{weather_crime_daily <-}\StringTok{ }\NormalTok{crime_daily }\OperatorTok\StringTok{ }
\StringTok{  }\KeywordTok{left_join}\NormalTok{(weather_daily) }\OperatorTok\StringTok{ }
\StringTok{  }\KeywordTok{filter}\NormalTok{(}\OperatorTok{!}\KeywordTok{is.na}\NormalTok{(temp)) }\OperatorTok\StringTok{ }
\StringTok{  }\KeywordTok{left_join}\NormalTok{(neighborhood_cluster) }\OperatorTok\StringTok{ }
\StringTok{  }\KeywordTok{mutate}\NormalTok{(}\DataTypeTok{Month =} \KeywordTok{factor}\NormalTok{(Month), }\DataTypeTok{Day =} \KeywordTok{factor}\NormalTok{(Day))}
\end{Highlighting}
\end{Shaded}

\section{Buliding models}\label{buliding-models}

Because this is an \textbf{imbalanced dataset} with more 0s than 1s,
I'll use \textbf{AUC} (area under the curve) as the metric to determine
which model to use. AUC considers every possible threshold value for
classification, and can give you a general idea that how your model
performs. Other metrics to consider are precision, recall, and F index.
Here we will focus only on AUC. I built different models for different
types of crime, and for demonstration, I only included property crime,
which is the most common crime here.

\subsection{Identify near zero variance
variables}\label{identify-near-zero-variance-variables}

\begin{Shaded}
\begin{Highlighting}[]
\NormalTok{zero =}\StringTok{ }\KeywordTok{nearZeroVar}\NormalTok{(weather_crime_daily[,}\DecValTok{18}\OperatorTok{:}\DecValTok{52}\NormalTok{])}
\NormalTok{zeroIndex =}\StringTok{ }\NormalTok{zero }\OperatorTok{+}\StringTok{ }\DecValTok{17}
\NormalTok{weather_crime_daily =}\StringTok{ }\NormalTok{weather_crime_daily[,}\OperatorTok{-}\NormalTok{zeroIndex]}
\end{Highlighting}
\end{Shaded}

\subsection{Relevel crime factor so that Yes is treated as
positive}\label{relevel-crime-factor-so-that-yes-is-treated-as-positive}

\begin{Shaded}
\begin{Highlighting}[]
\NormalTok{weather_crime_daily}\OperatorTok{$}\NormalTok{Crime =}\StringTok{ }\KeywordTok{factor}\NormalTok{(weather_crime_daily}\OperatorTok{$}\NormalTok{Crime, }\DataTypeTok{levels =} \KeywordTok{c}\NormalTok{(}\StringTok{'Yes'}\NormalTok{,}\StringTok{'No'}\NormalTok{))}
\end{Highlighting}
\end{Shaded}

\subsection{Split to training set and test
set}\label{split-to-training-set-and-test-set}

For quick training and demonstration purpose, I only used 1 year of data
to train the model, and most recent 3 months as test set.

\begin{Shaded}
\begin{Highlighting}[]
\NormalTok{train_property <-}\StringTok{ }\NormalTok{weather_crime_daily }\OperatorTok\StringTok{ }\KeywordTok{filter}\NormalTok{(Date }\OperatorTok{<}\KeywordTok{ymd}\NormalTok{(}\DecValTok{20181001}\NormalTok{),Date }\OperatorTok{>=}\StringTok{ }\KeywordTok{ymd}\NormalTok{(}\DecValTok{20171001}\NormalTok{),Type }\OperatorTok{==}\StringTok{ 'PROPERTY'}\NormalTok{)}
\NormalTok{test_property <-}\StringTok{ }\NormalTok{weather_crime_daily }\OperatorTok\StringTok{ }\KeywordTok{filter}\NormalTok{(Date }\OperatorTok{>=}\StringTok{ }\KeywordTok{ymd}\NormalTok{(}\DecValTok{20181001}\NormalTok{),Type }\OperatorTok{==}\StringTok{ 'PROPERTY'}\NormalTok{)}
\KeywordTok{set.seed}\NormalTok{(}\DecValTok{32}\NormalTok{)}
\NormalTok{myFolds <-}\StringTok{ }\KeywordTok{createFolds}\NormalTok{(train_property}\OperatorTok{$}\NormalTok{Crime, }\DataTypeTok{k =} \DecValTok{3}\NormalTok{) }\CommentTok{# Three folds cross validation}
\NormalTok{myControl =}\StringTok{ }\KeywordTok{trainControl}\NormalTok{(}\DataTypeTok{summaryFunction =}\NormalTok{ twoClassSummary, }\DataTypeTok{classProbs =}\NormalTok{ T,}\DataTypeTok{verboseIter =}\NormalTok{ T,}
                          \DataTypeTok{savePredictions =}\NormalTok{ T,}
                          \DataTypeTok{index =}\NormalTok{ myFolds, }\DataTypeTok{trim =}\NormalTok{ T, }\DataTypeTok{returnData =}\NormalTok{ F}
\NormalTok{                         )}
\NormalTok{formula_train =}\StringTok{ }\KeywordTok{as.formula}\NormalTok{(}\StringTok{'as.factor(Crime)~ Month + Day + Weekday + temp + temp_min + temp_max + pressure + }
\StringTok{                   humidity + wind_speed + clouds_all + broken_clouds + few_clouds + fog + haze +}
\StringTok{                           heavy_intensity_rain + light_intensity_drizzle + light_rain + mist + moderate_rain+}
\StringTok{                           overcast_clouds +  scattered_clouds + sky_is_clear + cluster'}\NormalTok{)}
\end{Highlighting}
\end{Shaded}

\subsection{Different models}\label{different-models}

\subsubsection{Logistic regression with
regularization}\label{logistic-regression-with-regularization}

\begin{Shaded}
\begin{Highlighting}[]
\NormalTok{model_glmnet <-}\StringTok{ }\KeywordTok{train}\NormalTok{(formula_train, }\DataTypeTok{trControl =}\NormalTok{ myControl, }\DataTypeTok{tuneGrid =} \KeywordTok{expand.grid}\NormalTok{(}\DataTypeTok{alpha =} \KeywordTok{c}\NormalTok{(}\DecValTok{0}\NormalTok{,}\FloatTok{0.1}\NormalTok{,}\DecValTok{1}\NormalTok{), }\DataTypeTok{lambda =} \KeywordTok{c}\NormalTok{(}\DecValTok{0}\NormalTok{,}\FloatTok{0.5}\NormalTok{,}\DecValTok{1}\NormalTok{)),}
                      \DataTypeTok{method =} \StringTok{'glmnet'}\NormalTok{,}\DataTypeTok{metric =} \StringTok{'ROC'}\NormalTok{, }\DataTypeTok{data =}\NormalTok{ train_property)}
\end{Highlighting}
\end{Shaded}

\begin{verbatim}
## + Fold1: alpha=0.0, lambda=1 
## - Fold1: alpha=0.0, lambda=1 
## + Fold1: alpha=0.1, lambda=1 
## - Fold1: alpha=0.1, lambda=1 
## + Fold1: alpha=1.0, lambda=1 
## - Fold1: alpha=1.0, lambda=1 
## + Fold2: alpha=0.0, lambda=1 
## - Fold2: alpha=0.0, lambda=1 
## + Fold2: alpha=0.1, lambda=1 
## - Fold2: alpha=0.1, lambda=1 
## + Fold2: alpha=1.0, lambda=1 
## - Fold2: alpha=1.0, lambda=1 
## + Fold3: alpha=0.0, lambda=1 
## - Fold3: alpha=0.0, lambda=1 
## + Fold3: alpha=0.1, lambda=1 
## - Fold3: alpha=0.1, lambda=1 
## + Fold3: alpha=1.0, lambda=1 
## - Fold3: alpha=1.0, lambda=1 
## Aggregating results
## Selecting tuning parameters
## Fitting alpha = 1, lambda = 0 on full training set
## Final model footprint reduced by 71.6 Mb or 100%
\end{verbatim}

Let's plot the results

\begin{Shaded}
\begin{Highlighting}[]
\KeywordTok{plot}\NormalTok{(model_glmnet)}
\end{Highlighting}
\end{Shaded}

\includegraphics{BaltimoreCrimePrediction_files/figure-latex/unnamed-chunk-26-1.pdf}

\begin{Shaded}
\begin{Highlighting}[]
\KeywordTok{print}\NormalTok{(model_glmnet)}
\end{Highlighting}
\end{Shaded}

\begin{verbatim}
## glmnet 
## 
## No pre-processing
## Resampling: Bootstrapped (3 reps) 
## Summary of sample sizes: 33823, 33824, 33823 
## Resampling results across tuning parameters:
## 
##   alpha  lambda  ROC        Sens       Spec     
##   0.0    0.0     0.7590607  0.1377164  0.9807128
##   0.0    0.5     0.7154413  0.0000000  1.0000000
##   0.0    1.0     0.7116081  0.0000000  1.0000000
##   0.1    0.0     0.7595928  0.1655026  0.9745264
##   0.1    0.5     0.5992414  0.0000000  1.0000000
##   0.1    1.0     0.5000000  0.0000000  1.0000000
##   1.0    0.0     0.7596763  0.1656797  0.9743000
##   1.0    0.5     0.5000000  0.0000000  1.0000000
##   1.0    1.0     0.5000000  0.0000000  1.0000000
## 
## ROC was used to select the optimal model using the largest value.
## The final values used for the model were alpha = 1 and lambda = 0.
\end{verbatim}

Lambda = 0, alpha = 1 gives us the highest ROC. This means we don't need
regularization, the model didn't overfit the data.

\subsubsection{Linear discriminant
analysis}\label{linear-discriminant-analysis}

\begin{Shaded}
\begin{Highlighting}[]
\NormalTok{model_lda <-}\StringTok{ }\KeywordTok{train}\NormalTok{(formula_train, }\DataTypeTok{trControl =}\NormalTok{ myControl, }
                      \DataTypeTok{method =} \StringTok{'lda'}\NormalTok{,}\DataTypeTok{metric =} \StringTok{'ROC'}\NormalTok{, }\DataTypeTok{data =}\NormalTok{ train_property)}
\end{Highlighting}
\end{Shaded}

\begin{verbatim}
## + Fold1: parameter=none 
## - Fold1: parameter=none 
## + Fold2: parameter=none 
## - Fold2: parameter=none 
## + Fold3: parameter=none 
## - Fold3: parameter=none 
## Aggregating results
## Fitting final model on full training set
\end{verbatim}

Let's look at the results

\begin{Shaded}
\begin{Highlighting}[]
\KeywordTok{print}\NormalTok{(model_lda)}
\end{Highlighting}
\end{Shaded}

\begin{verbatim}
## Linear Discriminant Analysis 
## 
## No pre-processing
## Resampling: Bootstrapped (3 reps) 
## Summary of sample sizes: 33823, 33824, 33823 
## Resampling results:
## 
##   ROC        Sens       Spec     
##   0.7590269  0.2943871  0.9368636
\end{verbatim}

\subsubsection{Quadratic discriminant
analysis}\label{quadratic-discriminant-analysis}

\begin{Shaded}
\begin{Highlighting}[]
\NormalTok{model_qda <-}\StringTok{ }\KeywordTok{train}\NormalTok{(formula_train, }\DataTypeTok{trControl =}\NormalTok{ myControl, }
                      \DataTypeTok{method =} \StringTok{'qda'}\NormalTok{,}\DataTypeTok{metric =} \StringTok{'ROC'}\NormalTok{, }\DataTypeTok{data =}\NormalTok{ train_property)}
\end{Highlighting}
\end{Shaded}

\begin{verbatim}
## + Fold1: parameter=none 
## - Fold1: parameter=none 
## + Fold2: parameter=none 
## - Fold2: parameter=none 
## + Fold3: parameter=none 
## - Fold3: parameter=none 
## Aggregating results
## Fitting final model on full training set
\end{verbatim}

Let's look at the results

\begin{Shaded}
\begin{Highlighting}[]
\KeywordTok{print}\NormalTok{(model_qda)}
\end{Highlighting}
\end{Shaded}

\begin{verbatim}
## Quadratic Discriminant Analysis 
## 
## No pre-processing
## Resampling: Bootstrapped (3 reps) 
## Summary of sample sizes: 33823, 33824, 33823 
## Resampling results:
## 
##   ROC        Sens       Spec     
##   0.7437636  0.4307622  0.8690768
\end{verbatim}

\subsubsection{Random forest}\label{random-forest}

\begin{Shaded}
\begin{Highlighting}[]
\NormalTok{model_ranger <-}\StringTok{ }\KeywordTok{train}\NormalTok{(formula_train, }\DataTypeTok{trControl =}\NormalTok{ myControl, }\DataTypeTok{tuneGrid =} \KeywordTok{expand.grid}\NormalTok{(}\DataTypeTok{mtry =} \KeywordTok{c}\NormalTok{(}\DecValTok{2}\NormalTok{,}\DecValTok{4}\NormalTok{,}\DecValTok{10}\NormalTok{), }\DataTypeTok{splitrule =} \StringTok{'gini'}\NormalTok{, }\DataTypeTok{min.node.size =} \DecValTok{1}\NormalTok{),}
                      \DataTypeTok{method =} \StringTok{'ranger'}\NormalTok{,}\DataTypeTok{metric =} \StringTok{'ROC'}\NormalTok{, }\DataTypeTok{data =}\NormalTok{ train_property,}\DataTypeTok{importance=}\StringTok{'impurity'}\NormalTok{)}
\end{Highlighting}
\end{Shaded}

Let's plot the results

\begin{Shaded}
\begin{Highlighting}[]
\KeywordTok{varImp}\NormalTok{(model_ranger)}
\end{Highlighting}
\end{Shaded}

\begin{verbatim}
## ranger variable importance
## 
##   only 20 most important variables shown (out of 75)
## 
##                  Overall
## cluster7         100.000
## cluster9          74.411
## cluster10         28.095
## cluster4          27.958
## cluster6          27.500
## cluster3          16.724
## cluster2           5.803
## cluster8           4.511
## temp               3.902
## humidity           3.456
## temp_max           3.416
## temp_min           3.359
## pressure           3.356
## clouds_all         3.189
## sky_is_clear       2.395
## wind_speed         2.306
## broken_clouds      2.051
## mist               2.013
## scattered_clouds   1.976
## overcast_clouds    1.911
\end{verbatim}

\begin{Shaded}
\begin{Highlighting}[]
\KeywordTok{plot}\NormalTok{(model_ranger)}
\end{Highlighting}
\end{Shaded}

\includegraphics{BaltimoreCrimePrediction_files/figure-latex/unnamed-chunk-32-1.pdf}

\begin{Shaded}
\begin{Highlighting}[]
\KeywordTok{print}\NormalTok{(model_ranger)}
\end{Highlighting}
\end{Shaded}

\begin{verbatim}
## Random Forest 
## 
## No pre-processing
## Resampling: Bootstrapped (3 reps) 
## Summary of sample sizes: 33823, 33824, 33823 
## Resampling results across tuning parameters:
## 
##   mtry  ROC        Sens          Spec     
##    2    0.7268018  5.061241e-05  0.9999939
##    4    0.7137099  7.655127e-02  0.9872112
##   10    0.7007627  1.628960e-01  0.9582803
## 
## Tuning parameter 'splitrule' was held constant at a value of gini
## 
## Tuning parameter 'min.node.size' was held constant at a value of 1
## ROC was used to select the optimal model using the largest value.
## The final values used for the model were mtry = 2, splitrule = gini
##  and min.node.size = 1.
\end{verbatim}

\subsubsection{Gradient boosting}\label{gradient-boosting}

\begin{Shaded}
\begin{Highlighting}[]
\NormalTok{model_gbm <-}\StringTok{ }\KeywordTok{train}\NormalTok{(formula_train, }\DataTypeTok{trControl =}\NormalTok{ myControl, }\DataTypeTok{tuneGrid =} \KeywordTok{expand.grid}\NormalTok{(}\DataTypeTok{shrinkage =} \KeywordTok{c}\NormalTok{(}\FloatTok{0.01}\NormalTok{,}\FloatTok{0.1}\NormalTok{,}\FloatTok{0.2}\NormalTok{), }\DataTypeTok{interaction.depth =} \KeywordTok{c}\NormalTok{(}\DecValTok{1}\NormalTok{,}\DecValTok{2}\NormalTok{,}\DecValTok{4}\NormalTok{), }\DataTypeTok{n.minobsinnode =} \DecValTok{10}\NormalTok{, }\DataTypeTok{n.trees =} \DecValTok{1000}\NormalTok{),}
                      \DataTypeTok{method =} \StringTok{'gbm'}\NormalTok{,}\DataTypeTok{metric =} \StringTok{'ROC'}\NormalTok{, }\DataTypeTok{data =}\NormalTok{ train_property)}
\end{Highlighting}
\end{Shaded}

Let's plot the results

\begin{Shaded}
\begin{Highlighting}[]
\KeywordTok{varImp}\NormalTok{(model_gbm)}
\end{Highlighting}
\end{Shaded}

\begin{verbatim}
## gbm variable importance
## 
##   only 20 most important variables shown (out of 75)
## 
##                Overall
## cluster8      100.0000
## cluster10      47.0639
## cluster3       46.5467
## cluster5       23.2885
## cluster4       20.6189
## cluster6       17.7766
## cluster2       16.1516
## cluster7        9.7990
## cluster9        7.9888
## temp            3.0465
## pressure        2.7730
## humidity        2.5948
## clouds_all      2.3865
## temp_max        1.7916
## Month10         1.3736
## WeekdaySun      1.3425
## WeekdaySat      1.1367
## Month3          1.1016
## broken_clouds   1.0808
## temp_min        0.9621
\end{verbatim}

\begin{Shaded}
\begin{Highlighting}[]
\KeywordTok{plot}\NormalTok{(model_gbm)}
\end{Highlighting}
\end{Shaded}

\includegraphics{BaltimoreCrimePrediction_files/figure-latex/unnamed-chunk-34-1.pdf}

\begin{Shaded}
\begin{Highlighting}[]
\KeywordTok{print}\NormalTok{(model_gbm)}
\end{Highlighting}
\end{Shaded}

\begin{verbatim}
## Stochastic Gradient Boosting 
## 
## No pre-processing
## Resampling: Bootstrapped (3 reps) 
## Summary of sample sizes: 33823, 33824, 33823 
## Resampling results across tuning parameters:
## 
##   interaction.depth  ROC        Sens       Spec     
##   1                  0.7517245  0.1595050  0.9748446
##   2                  0.7488788  0.1788136  0.9680708
##   4                  0.7444652  0.1852414  0.9635916
## 
## Tuning parameter 'n.trees' was held constant at a value of 1000
## 
## Tuning parameter 'shrinkage' was held constant at a value of 0.1
## 
## Tuning parameter 'n.minobsinnode' was held constant at a value of 10
## ROC was used to select the optimal model using the largest value.
## The final values used for the model were n.trees = 1000,
##  interaction.depth = 1, shrinkage = 0.1 and n.minobsinnode = 10.
\end{verbatim}

\subsection{Compare all the models}\label{compare-all-the-models}

\begin{Shaded}
\begin{Highlighting}[]
\NormalTok{model_all =}\StringTok{ }\KeywordTok{list}\NormalTok{(}\DataTypeTok{glmnet =}\NormalTok{ model_glmnet, }\DataTypeTok{lda =}\NormalTok{ model_lda, }\DataTypeTok{qda =}\NormalTok{ model_qda, }\DataTypeTok{ranger =}\NormalTok{ model_ranger,}
                 \DataTypeTok{gbm =}\NormalTok{ model_gbm)}
\NormalTok{results =}\StringTok{ }\KeywordTok{resamples}\NormalTok{(model_all)}
\KeywordTok{summary}\NormalTok{(results)}
\end{Highlighting}
\end{Shaded}

\begin{verbatim}
## 
## Call:
## summary.resamples(object = results)
## 
## Models: glmnet, lda, qda, ranger, gbm 
## Number of resamples: 3 
## 
## ROC 
##             Min.   1st Qu.    Median      Mean   3rd Qu.      Max. NA's
## glmnet 0.7582697 0.7588491 0.7594285 0.7596763 0.7603797 0.7613308    0
## lda    0.7575681 0.7582330 0.7588980 0.7590269 0.7597564 0.7606148    0
## qda    0.7432933 0.7436039 0.7439145 0.7437636 0.7439987 0.7440830    0
## ranger 0.7243869 0.7254977 0.7266084 0.7268018 0.7280092 0.7294101    0
## gbm    0.7496629 0.7507765 0.7518900 0.7517245 0.7527553 0.7536206    0
## 
## Sens 
##             Min.   1st Qu.    Median         Mean      3rd Qu.
## glmnet 0.1559368 0.1588217 0.1617067 1.656797e-01 1.705512e-01
## lda    0.2935014 0.2935393 0.2935773 2.943871e-01 2.948299e-01
## qda    0.4271940 0.4279912 0.4287883 4.307622e-01 4.325463e-01
## ranger 0.0000000 0.0000000 0.0000000 5.061241e-05 7.591862e-05
## gbm    0.1535074 0.1547981 0.1560887 1.595050e-01 1.625038e-01
##                Max. NA's
## glmnet 0.1793956878    0
## lda    0.2960825995    0
## qda    0.4363042818    0
## ranger 0.0001518372    0
## gbm    0.1689189189    0
## 
## Spec 
##             Min.   1st Qu.    Median      Mean   3rd Qu.      Max. NA's
## glmnet 0.9722809 0.9737584 0.9752359 0.9743000 0.9753095 0.9753832    0
## lda    0.9358054 0.9365030 0.9372006 0.9368636 0.9373927 0.9375849    0
## qda    0.8667646 0.8678556 0.8689467 0.8690768 0.8702328 0.8715190    0
## ranger 0.9999816 0.9999908 1.0000000 0.9999939 1.0000000 1.0000000    0
## gbm    0.9727214 0.9740429 0.9753644 0.9748446 0.9759062 0.9764479    0
\end{verbatim}

\begin{Shaded}
\begin{Highlighting}[]
\KeywordTok{dotplot}\NormalTok{(results)}
\end{Highlighting}
\end{Shaded}

\includegraphics{BaltimoreCrimePrediction_files/figure-latex/unnamed-chunk-35-1.pdf}

Logistic regression model wins, let's see how it performs on test set

\subsection{Generalization}\label{generalization}

\begin{Shaded}
\begin{Highlighting}[]
\NormalTok{roc_obj <-}\StringTok{ }\KeywordTok{roc}\NormalTok{(test_property}\OperatorTok{$}\NormalTok{Crime, }\KeywordTok{predict}\NormalTok{(model_glmnet, test_property, }\DataTypeTok{type =} \StringTok{'prob'}\NormalTok{)}\OperatorTok{$}\NormalTok{Yes)}
\KeywordTok{auc}\NormalTok{(roc_obj)}
\end{Highlighting}
\end{Shaded}

\begin{verbatim}
## Area under the curve: 0.7619
\end{verbatim}

\section{Conclusion}\label{conclusion}

Logistic regression has the highest AUC, while random forest trained
using `ranger' package has the lowest AUC. This means the relationship
between the log(odds) and our predictor variables are linearly related
rather than have very complex interactions. However, if we choose
different metric such as sensitivity, qda would give us the highest
accuracy. However, the specificity, which is the true negative rate, is
also the lowest using qda. This means we classify more false positives.
It all depends on what your \textbf{goal} is and you have to
\textbf{comporise} between sensitivity and specificity.

Besides property crime, I also built separate models for other five
types of crime. Building models is not the end of my project. For better
\textbf{visualization}, I deployed a shiny web application where people
can check future four days of every type of crime probability at every
neighborhood in Baltimore based on real-time weather forecast.
\href{https://xuyangjhu.shinyapps.io/BaltimoreCrime/}{Please check it
out}


\end{document}
